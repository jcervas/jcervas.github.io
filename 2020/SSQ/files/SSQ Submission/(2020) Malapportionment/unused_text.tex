=================================================================
% -- SECTION -- SECTION -- SECTION -- SECTION -- SECTION -- SECTI%
% =================================================================
\section{Conclusion} \label{sec:conclusion}
% =================================================================
% 
Unlike other types of inequities, malapportionment does have a normative claim of a \textit{fair} distribution: one person, one vote requires that each vote be equally weighted. When choosing how to measure malapportionment, we should be careful about overstating (or understating) the deviations from this standard. \citet[][]{Monroe1994}, in his brilliant essay on the subject, suggests that fair and inequity are distinct concepts: ``Since all electoral systems are disproportional to some extent, and all distributions of geographical representation are malapportioned to some extent, this does not appear empirically to be a generally accepted equivalence...''. 

Our purpose in this essay isn't to identify which measures adhere to desired axioms but rather to take a number of identifiable measures of inequity and empirically evaluate three different U.S. political institutions over time. We find that some measures so similar patterns over time, and all give the same answer to the question about which of the institutions we measure are the most malapportioned. The U.S. Senate is, by far, the most malapportioned institution in the U.S., and has been since the founding. Unlike the U.S. House and the Electoral College, the Senate has increased in it's malapportionment since the founding of the republic. We conclude by noting that the court measures of malapportionment (TPD, Max/Min) which look at just two states might be sufficient for determining if any malapportionment exists (deviation from perfect one person, one vote), but are unsatisfactory to determing the \textit{level} of malapportionment. The political science measures (Gallagher, Loosemore-Hanby) do a better job of summarizing the amount an institution deviates from equity, but have conflicting conclusions about the level of malapportionment over time as determined by their correlation with each other. Finally, the economist's measures (Gini coefficient, Fractiles, Minimum population) take into account the entire distribution of voting power, and consider the individuals which make up that distribution. They also provide for a satisfying empirical estimate of the malapportionment, giving either a percentage of inequality (gini), the percent of power held by a percentile of people (fractiles), and the minimum amount of the population that can control the institution (minimum population). The latter of these measures also provides a satisfactory empirical measure of majoritarianism, which is the property that most people appear concerned about 


 JONATHAN DELETE THE NEXT SENTENCE. BECAUSE SOME OF THE PATTERNS ARE NON-MONOTONIC, IT REALLY IS NOT CORRECT. The greatest discrepancy between values for the House and Senate occurs using the \textit{Gini} \textit{coefficient}, and the next largest gap using minimum proportion needed to control a majority of seats, with the lowest gap arising from use of the \textit{total population deviation}. END DELETION
 
 DELETE THE NEXT SENTENCES AS EITHER REDUNDANT OR CONFUSING. We believe the best explanation for this is based on population changes across states of different sizes. We can see this by looking at the deviations from ideal for all the units with populations below the ideal. Despite the shifting number of seats in each of the institutions (or freezing of the House seats), the population of units are similar relative to the ideal. This is true too of the deviations from ideal for those units above ideal. END DELETION.
\lettrine[lines=5]{T}{he} United States spent \$13 billion in 2010 conducting the Census. In 2020, Congress has allocated no additional resources towards counting the population, forcing the Commerce Department to find cost-cutting measures while still achieving a secure and accurate count.
    % ================================================================= 
        % ------- FOOTNOTE --------------------------------------------
            \footnote{D. Sunshine Hillygus, personal communication, January 27, 2020.}
        % ------- END FOOTNOTE ----------------------------------------
    % =================================================================   
This constitutionally mandated count determines how the 435 members of Congress are distributed among the states, and determines the relative weight of each state in the Electoral College. Putting aside the difficulty in accurately counting a geographically diverse country, all apportionment formulas result in malapportionment; a difference between the number of voter in each legislative district and expected number of voters based on the total population divided by the total number of districts. Indeed, in every democratic society where voters elect representatives creates some discernible malapportionment. Single-member districts under rules of first-past-the-post tend to give a ''winner's bonus'' to the party that wins the majority. 

Even the most equal of parliaments, utilizing rules of proportionality, have rules for rounding of votes and thresholds for obtaining seats that creates a gap between the votes cast for a party and the seats it is assigned from those votes. Malapportionment, in and of itself, may or may not have direct pernicious consequences for the treatment of particular political parties or cognizable groups of voters with distinct interests. For example, Singapore has high levels of malapportionment, but that malapportionment does not appear to have effects that favor the ruling party, the PAP \citep{Tan2018}. In contrast, malapportionment in Japan has historically favored rural areas by over-representing rural voters, and thus been a boon to the LDP whose greatest strength came from rural voters \bg{(BG FILL IN REF)}. In the U.S., although malapportionment bias is often regarded as inherently undesirable (normative), %

            % \textit{Connor v. Finch}, 431 U.S. 407 (1977). In situations where the court provides remedial maps ``the court's task is inevitably an exposed and sensative one that must be accomplished circumspectly, and in a manner `free from any taint of arbitrariness or discrimination''' (quoting Roman v. Sincock, 377 U.S. 695, 710 (1964)).

            \footnote{A TPD above 10\% would be permitted only if the jurisdiction could offer legitimate (and compelling) reasons for the deviation. In \textit{Mahan v. Howell}, 410 U.S. 315 (1973), the court upheld Virginia State House districts that had a total population deviation of 16.4\% on the grounds that it was ``based on legitimate considerations incident to the effectuation of a rational state policy'' \textit{Reynolds v. Sims}, 377 U. S., at 579.}  But, in the light of subsequent court decisions it is unlikely that this level of population discrepancy would now be tolerated for state legislative districts.}

 The first approach measures ``moments'' in a distribution, which can be shown in Lorenz curves \citep{Butler1987}. A Lorenz plot measures the proportion of all income at different percentiles of the population. For instance, we could construct a Lorenz curve of all Americans and find the proportion of the total income for individuals at the 20 and 80 percentile. . Constructed this way, a society with a dictator means that one individual (at the 100 percentile) holds 100\% of the voting power. We could compare ratios of different percentiles, say 20\% and 80\%, and compare across apportionment. This ratio approach, like that of looking at the largest and smallest states (equation \ref{eq:maxmin}), risks missing the nuance that would make a distribution of power appear more equal when taking into account the entire distributions. We non-the-less report the 80/20 ratios in Figure \new{X}.
 
 Of course, we do not have the same expectations (or moral judgments) about income inequality as we do about representational inequality because democratic theory requires our political institutions to promote equality but capitalist institutions do not give rise to the same expectations of equality
 
 
 
 The next measure which has played a central role in the inequality literature is the \textit{Gini Index} \citep{Gastwirth1972}. IA Gini Index measure of \textit{a priori} inequity in apportionment based on a Lorenz curve, parallels a standard way of looking at inequity in income distribution \citep[see e.g.,][]{BaiLagunoff2013}. We create such an index by arranging the states from most underrepresented to most overrepresented in terms and then plotting the cumulative distribution of a state's vote share against the cumulative distribution of the population vote share for those same states, to give us a Lorenz curve. The Gini index is the area between an Lorenz curve and a 45 degree line \citep{Dalton1920, Yntema1933,Sen1976, Yitzhaki1983}. As such, it can be interpreted as a percentage of inequality. When a dictator controls all the voting power, the maximum vote inequality, the ratio is 1; the entire area below the line of equality is included. This index allows for easy interpretation, generalizes to related concepts \citep{Sen1976, Yitzhaki1979}, has population independence such that the size of the unit of analysis doesn't affect the coefficient \citep{Gastwirth2017}, and gives us a measure satisfying Dalton's principle of transfers \citep{Dalton1920}. The downfall of this measure is it fails to distinguish the distribution of inequality. In economics, specifically in the study of inequality, this feature is particularly difficult because there is no normative basis to think on distribution is better than another, except in the specific case of socialism. Here, there is a democratic norm in which any distribution can be compared to, that of proportional representation. Only proportional representation satisfies the \textit{one person, one vote} principle, and regardless of how vote-power is distributed, if it deviates from proportionality, can can be said to be an abstraction from democratic principles.
 JONATHAN, DELETE THEfootnote  BELOW AS irrelevant
    % ================================================================= 
    % -- FOOTNOTE -- FOOTNOTE -- FOOTNOTE -- FOOTNOTE -- FOOTNOTE --  %
    % -----------------------------------------------------------------
        \footnote{See \citet[][p. 316]{Browning_King_1987_seats_votes} for why Gini index is an inappropriate measure of partisan bias. There, the authors are measuring votes modelled in a bilogit form, which is functionally different than our conception of inequality.}
    % ----------------------------------------------------------------- 
    % -- END FOOTNOTE -- END FOOTNOTE -- END FOOTNOTE -- END FOOTNOTE %
    % =================================================================
\begin{quote}
  Ratios of largest-to-smallest districts might seem an obvious means for assessing malapportionment. However, such ratios actually prove poor indicators of malapportionment. First, district size on the basis of population tells us little about the degree to which districts are under-represented or over-represented: we also need to know how many seats are allocated to each district. Furthermore, even if we know how many seats are held by the largest and smallest districts and can therefore calculate ratios of 'worst represented' to 'best represented' districts, such ratios tell us little about overall degrees of malapportionment. For example, even if this ratio is 50:1 (e.g., a single-member district system in which the largest district has a population fifty times greater than the smallest district), all other districts may have nearly-equivalent populations, and, hence, the largest and smallest districts could be extreme outliers in a system with a low degree of average malapportionment. Although it may be tempting to interpret wide gaps between the best and worst represented districts as signs of high overall levels of inequality in electoral systems, a better measure is required. \dots[T]he Loosemore-Hanby index of electoral disproportionality (D) provides such a measure (\textit{internal footnotes omitted}).
 \end{quote}
 
 
        \textit{Baker v. Carr } TO constitutionally required textit{Baker v. Carr} established that malapportionment claims under the Equal Protection Clause are justiciable, but did not say what the standard would be. In \textit{Reynolds v. Sims} 377 U.S. 533 (1964), Alabama voters challenged unequally populated state legislative districts as violating  the Equal Protection Clause of the 14th Amendment. The court ruled that "one person, one vote" is constitutionally required. .  
        
            \footnote{The justification of the higher standard for congressional districting was that it was rooted in Article I defining the role of the House as the repository of the popular principle of representation, while justification for state and local malapportionment standards were found in the Equal Protection clause of the $14^{th}$ Amendment (see also \textit{Reynolds v. Sims} § VI). The court has required legitimate justifications even for \textit{de minimis} population deviations below the 10\% threshold that applies to state legislative elections \citep[][see also  \textit{Wesberry v. Sanders}, 376 U. S. 1 (1964); \textit{Kirkpatrick v. Preisler}, 394 U. S. 526 (1969); \textit{Wells v. Rockefeller}, 394 U. S. 542 (1969); \textit{White v. Weiser}, 412 U. S. 783 (1973); \textit{Gaffney v. Cummings}, 412 U. S. 735 (1973); \textit{White v.Regester} 412 U.S. 755 (1973)]{electionlaw2018}.}
            
            ================================================================= % ------- FOOTNOTE -------------------------------------------- \footnote{For example, \citet{Dahl1971} regards the one person, one vote principle as a necessary component of democratic governance. \citet{Taagepera1989} has called malapportionment a pathology.} % ------- END FOOTNOTE ---------------------------------------- % ================================================================= it was the perceived effects on government policies stemming from under-representation of city dwellers that motivated much of the sentiment that agitated pre-Baker v. Carr reformers (see e.g. \textit{Baker v. Carr}, 369 U.S. 186, 1962, \textit{Reynolds v. Sims}, 377 U.S. 533, 1964. % ================================================================= % ------- FOOTNOTE -------------------------------------------- \footnote{\citet[][pg. 653]{Samuels2001}, reviewing a number of single country studies, concludes ``malapportionment can have an important impact on executive-legislative relations, intra-legislative bargaining and the overall performance of democratic systems.''} % ------- END FOOTNOTE ---------------------------------------- % ================================================================= But, regardless of why we regard malapportionment as problematic, we must first answer the question: `How do we measure malapportionment?
   
   
   Moreover, we also show that, for both the House and the EC, the historical time series data on the magnitude of the malapportionment is very flat regardless of which measure we use, with \bg{two} measures even showing a very minor downtrend in recent elections.
   
   
   
   
  
\par
    \new{\citet{GrofmanScarrow1981} offer seven criteria by which apportionment can be judged. 
    % =================================================================
        % ------- QUOTE --------------------------------------------
        \begin{quote}
            (1) per capita equity in representation; (2) equality of citizen power to affect election outcomes; (3) equality of voter efficacy; (4) proportionate equality of legislators' power; (5) equality of citizen power to affect legislative outcomes; (6) equal numbers of legislators for equal numbers of citizens; (7) majority control of legislative outcomes. 
        \end{quote}
        % ------- END QUOTE -------------------------------------------
    % =================================================================
This essay focuses on criteria one, two, five, six, and seven. Criterion one and two are related to the premise brought forth in the court ruling in \textit{Reynolds v. Sims} (377 U.S. 560, 1964): ``\dots the fundamental principle of representative government in this country is one of equal representation for equal numbers of people\dots''. This is the principle of `one person, one vote'. This requires that all legislative districts be composed of the same numbers of individuals (distinct from equal number of voters, or eligible voters, or some other measure of equality). Criteria three is more closely related to the problems associated with gerrymandering. Criteria four is not applicable to U.S. national legislatures since legislators all have equal power of their votes. Criterion five and six is of particular focus for those that argue that the institutions provide some voters with the ability to affect outcomes in disproportionate manners. This is highlighted when the Electoral College delivers the presidency to a candidate who receives less votes than the loser. Criteria seven combines all the previous ones to determine if a majority of the people can control a majority of the institutions. Chief Justice Earl Warren expresses this opinion in \textit{Reynolds V. Sims}, 377 U.S. 533, 1964, earlier recognized by \citet{Dahl1956}:
    % =================================================================
        % ------- QUOTE --------------------------------------------
        \begin{quote}
            Logically, in a society ostensibly grounded on representative government, it would seem reasonable that a majority of the people of a State could elect a majority of that State's legislators. To conclude differently, and to sanction minority control of state legislative bodies, would appear to deny majority rights in a way that far surpasses any possible denial of minority rights that might otherwise be thought to result. Since legislatures are responsible for enacting laws by which all citizens are to be governed, they should be bodies which are collectively responsive to the popular will. And the concept of equal protection has been traditionally viewed as requiring the uniform treatment of persons standing in the same relation to the governmental action questioned or challenged. With respect to the allocation of legislative representation, all voters, as citizens of a State, stand in the same relation regardless of where they live. Any suggested criteria for the differentiation of citizens are insufficient to justify any discrimination, as to the weight of their votes, unless relevant to the permissible purposes of legislative apportionment. Since the achieving of fair and effective representation for all citizens is concededly the basic aim of legislative apportionment, we conclude that the Equal Protection Clause guarantees the opportunity for equal participation by all voters in the election of state legislators. Diluting the weight of votes because of place of residence impairs basic constitutional rights under the Fourteenth Amendment just as much as invidious discriminations based upon factors such as race, or economic status.
        \end{quote}}
        % ------- END QUOTE -------------------------------------------
    % =================================================================


    =================================================================
        ------- FOOTNOTE --------------------------------------------
         \footnote{We then show that, nonetheless, those who think the Electoral College is badly malapportioned will be found to be correct if we judge malapportionment by the usual metrics used to judge compliance with “one person, one vote” in contemporary U.S. voting rights law, but only because the U.S, House, viewed nationally and not state by state, is malapportioned by these same measures. In contrast, other metrics lead to a very different conclusion. When we evaluate EC malapportionment using the two metrics that are most common is the electoral systems literature to measure seats-votes disproportionality, namely the Gallagher Index (Gallagher, 1991) and the Loosemore-Hanby Index (Loosemore and Hanby, 1971), this analysis leads us to the conclusion that the EC behaves with respect to seats to population comparisons much like a proportional representation system does with respect to seats to votes comparisons. Similarly, Gini-index based measures of EC malapportionment suggest very little bias, especially as compared to the vast discrepancies we observe when we look at income distributions.}
        ------- END FOOTNOTE ----------------------------------------
    =================================================================
    
    
    
    When another state is roughly the same size as the smallest state peculiarities of the rounding process lead to overrepresentation of the larger of the two states. For instance, in 1790, Delaware had only a slightly lower population than Rhode Island, but due to the rounding formula used in apportionment, Rhode Island had one more congressional seat and thus one additional Electoral College vote, and was more overrepresented in the Electoral College than Delaware, even though Delaware was the smallest state at the time. After 1810 (except in 1830 and 1850), New York had the largest population but the state with the largest population often was not the most underrepresented in the EC relative to population because of the same type of rounding complication. This happened in 1800, 1810, 1820, 1830, 1840, 1850, 1860, 1880, 1920, and 1950, while in 1960 there were two states that were more underrepresented than New York. California surpassed New York as the largest state in 1970, but in 1970 it was not the most underrepresented state. 
    
    
To construct the Lorenz Curve, we find the share of the vote for each district. That is then divided but the number of people in each district. We order these such that the most under-represented are first and the most over-represented are last, and plot the cumulative share of the vote against the cumulative share of the population. We then find a function to estimate the best fitting curve on these points.
    % ================================================================= 
    % -- FOOTNOTE -- FOOTNOTE -- FOOTNOTE -- FOOTNOTE -- FOOTNOTE --  %
    % -----------------------------------------------------------------
        \footnote{We could alternatively find an approximate Lorenz curve by taking each district's $(j)$ share of the total vote $S$. We would then have $(n)$ districts of unequal populations. We can adjust the populations by weighting them by unit. $1 = \sum{\omega_{ij} \frac{1}{\sum{ni}}}$. Correctly specifying the weighted population and vote-share leads to no loss of accuracy.}
    % ----------------------------------------------------------------- 
    % -- END FOOTNOTE -- END FOOTNOTE -- END FOOTNOTE -- END FOOTNOTE %
    % =================================================================
    
        % ================================================================= 
    % -- FOOTNOTE -- FOOTNOTE -- FOOTNOTE -- FOOTNOTE -- FOOTNOTE --  %
    % -----------------------------------------------------------------
        \footnote{We need not estimate the coefficient for the entire population, because a fraction of the population with sufficient size, say 1/1000 is plenty to derive an estimate within four significant digits.}
    % ----------------------------------------------------------------- 
    % -- END FOOTNOTE -- END FOOTNOTE -- END FOOTNOTE -- END FOOTNOTE %
    % =================================================================
    
    % ================================================================= 
    % -- FOOTNOTE -- FOOTNOTE -- FOOTNOTE -- FOOTNOTE -- FOOTNOTE --  %
    % -----------------------------------------------------------------
        \footnote{The same is true for a seventh metric, the Gini Index. For space reasons we do not present the \textit{Gini Index} results here. They are available upon request from the authors.}
    % ----------------------------------------------------------------- 
    % -- END FOOTNOTE -- END FOOTNOTE -- END FOOTNOTE -- END FOOTNOTE %
    % =================================================================
    
To explain this puzzle we need to recognize that population growth patterns strongly affect malapportionment. We can account for the decline or flatness in our various malapportionment measures in recent decades by calling attention to two population growth phenomena.  On the one hand, the largest states have grown larger relative to the average state; thus, underrepresentation at the high end has been reduced since the larger states contain more of the total population.  On the other side of the coin, the smallest state has grown at roughly the same rate as the mean size of Electoral College seats has grown, so the smallest state is no further away from ideal population equality now than in the recent past, and thus the consequences of overrepresentation of small states is thus reduced or kept essentially constant. 
-- FOOTNOTE -- FOOTNOTE -- FOOTNOTE -- FOOTNOTE -- FOOTNOTE --  %
-----------------------------------------------------------------
    \footnote{In 1790, if we order states according to overrepresentation, 66.7\% of the states (with 44.2\% of the population) were needed to reach 50\% of the Electoral college; after the 2010 census, 78.4\% of the states (with 43.7\% of the population) are needed. In fact, the proportion of states in the minimum winning coalition has steadily increased since the founding (data omitted for space reasons). These time trends are shown in Figure 6.}
-----------------------------------------------------------------
-- END FOOTNOTE -- END FOOTNOTE -- END FOOTNOTE -- END FOOTNOTE %
=================================================================